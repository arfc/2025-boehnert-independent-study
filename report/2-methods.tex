\pagebreak
\section{Methodology}
\label{sec:methods}
The geometry of the lattice structure used in the two simulations is simple, consisting of a 2x2 2D lattice structure, in which three cells contain a fuel pin in a water moderator, the fourth cell containing only the water moderator.

Figure here

Two methods used to solve for neutron transport in this report are the Monte Carlo method and the Random Ray method. As previously mentioned, the Monte Carlo method run by code from OpenMC solves for transport stochastically, whereas the Random Ray method is a hybrid stochastic-deterministic method.

\pagebreak
\subsection{Comparison}
Table comparing flux and nu-fission: 8 rows 5 columns
\begin{figure}[h]
        \centering
\begin{tikzpicture}[node distance=1.5cm]
\node (start) [object] {Start};
\node (step1) [process, below of=start] {Step 1};
\node (intermediate) [object, below of=step1] {Intermediate};
\node (step2) [process, below of=intermediate] {Step 2};
\node (end) [object, below of=step2]{End};

\draw [arrow] (start) -- (step1); 
\draw [arrow] (step1) -- (intermediate);
\draw [arrow] (intermediate) -- (step2); 
\draw [arrow] (step2) -- (end);
\end{tikzpicture}
\caption{A caption for the flowchart.}
\label{fig:comp}
\end{figure}

\subsection{Shannon Entropy}
Figures of Shannon Entropy
