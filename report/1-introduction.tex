\section{Introduction}
\label{sec:intro}
This is the introduction section. For consistency, use the
\texttt{glossaries} package for acronyms such as ``\gls{SFR}".
This automatically populates the glossaries list and abbreviates future
mentions of ``\gls{SFR}".

All tables and figures must be referenced to in the text.
Use the BibTeX package to manage your paper references and bibliography
\cite{huff_extensions_2014}.

There are two main methods of computation within reactor physics that are commonly used to solve for neutron transport, those being simulations which employ Monte Carlo code from OpenMC, and simulations which employ the Random Ray Method. The Monte Carlo method makes use of random sampling to stochastically solve for neutron transport. As opposed to solving a complex transport equation, this method instead tracks the histories of many individual particles as they interact with each other, and assigns weights to them in order to streamline computation. OpenMC assigns each particle an angular direction and energy, both of which will correspond to the weight given, which changes as particles absorb or scatter. The Random Ray method is similarly a hybrid stochastic-deterministic  method for solving for neutron transport, which randomly selects characteristic lines for which particles follow and then deterministically solves the neutron transport equation. The key differences between Monte Carlo and Random Ray is that, instead of tracking individual particles, each ray in the Random Ray method represents one of these characteristic lines upon which the transport equation can be written and solved for analytically.

Both of these methods are key to solving for neutron transport in complex geometries. Though the geometry of the simulations produced in this case are not very complex, when considering full scale reactors, it is important to be able to accurately predict the behavior of neutron populations during reactor processes. By utilizing stochastic methods to evaluate neutrons---individual particles for Monte Carlo and rays for Random Ray---one can gain a more detailed understanding of variables within the reactor such as power densities and potential hotspots. In this case, the two methods are used to find the flux in various regions of the lattice structure, the fission rates of the individual fuel pins, and the Shannon Entropy.
